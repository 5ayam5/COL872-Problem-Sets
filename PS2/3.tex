\begin{solution}{Question 3: Impossibility of two-round zero-knowledge protocols with auxiliary information}\label{ques:3}
    \begin{question}
        Prove that two-round protocols (where the verifier sends the first message and prover sends the second message) cannot satisfy the zero-knowledge property in the presence of auxiliary information. Describe the argument in full detail.
    \end{question}
    \tcblower{}
    \begin{proof}
        (Proof Idea: We use the auxiliary input as the string that is sent to the prover. Thus, the simulator (even if it is non-black box) cannot "generate" a valid response unless L itself is in BPP)\par
        The default interaction between the prover and the verifier is defined as:
        \begin{enumerate}
            \item $V_1(x, r) = m_1$ is sent to the prover
            \item $P_2(x, m_1) = m_2$ is set to the verifier
            \item $V_3(m_1, m_2) = m_3 \in \{0, 1\}$ is the result of the interaction (rejected or accepted)
        \end{enumerate}
        Now consider the following verifier $V^* = (V_1^*, V_3^*)$ that takes in auxiliary information $z$ and interacts with the prover as follows:
        \begin{equation}
            V_1^*(z)(x, r) = z, V_3^*(z)(x, r, m_2) = V_3(m_1, m_2)
        \end{equation}
        Since by our assumption, the given protocol is two-round zero-knowledge in the presence of auxiliary information, we will have a simulator for $V^*$. Let any such simulator be $S^*$. We now propose the following algorithm $\calA$ for checking if $x \in \calL_{yes}$,
        \protocol{Algorithm $\calA$}{Algorithm for checking $x \in \calL_{yes}$}{lyes}{
            \begin{enumerate}
                \item Sample a random $r$ and compute $m_1 = V_1(x, r)$
                \item Obtain the transcript $(m_1, m_2, m_3)$ on running $S^*(x, m_1)$.
                \item If $S^*$ fails to simulate, output $0$. Else output $m_3$.
            \end{enumerate}
        }
        We first note that $\calA$ is a ppt algorithm since it uses $V_1, S^*$ in sequence which are both ppt algorithms.\par

        Now, consider the probability of $\calA$ in deciding $x \in \calL_{yes}$ when $x$ is a yes instance (assuming that completeness of the protocol is $c$ and soundness error is $s$),
        \begin{equation}
            \begin{split}
                \prob{\calA\text{ outputs }1} &= \prob{S^*\text{ does not output }\perp}\cdot \prob{m_3 = 1 | x \in \calL_{yes}}\\
                                             &= (1 - \mu(n))\cdot \prob{m_3 = 1 | x \in \calL_{yes}}\\
                                             &= (1 - \mu(n))\cdot c
            \end{split}
        \end{equation}
        Therefore, we can see that $\calA$ can decide $x \in \calL_{yes}$ with probability $ > 2/3$ in ppt.

        Again, consider the probability of $\calA$ in deciding $x \in \calL_{no}$ when $x$ is a no instance,
        \begin{equation}
            \begin{split}
                \prob{\calA\text{ outputs }0} &= \prob{S^*\text{ does not output }\perp}\cdot \prob{m_3 = 0 | x \in \calL_{no}}\\
                                             &= (1 - \mu(n))\cdot \prob{m_3 = 0 | x \in \calL_{no}} + \mu(n)\\
                                             &= (1 - \mu(n))\cdot (1 - s) + \mu(n)
            \end{split}
        \end{equation}
        Therefore, we can see that $\calA$ can decide $x \in \calL_{no}$ with probability $ > 2/3$ in ppt.

        Therefore, $\calL$ is in BPP.
    \end{proof}
\end{solution}
