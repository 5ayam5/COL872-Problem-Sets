\begin{solution}{HVZK PoK for DDH}\label{ques:2.2}
    \begin{question}
    Consider the following language:
    \begin{equation}
        \lddh = \{(g,h,u,v): \exists a \in \z_q \text{ s.t. }u = g^a , v = h^a \}
    \end{equation}
    Construct an honest-verifier zero-knowledge proof-of-knowledge protocol for $\lddh$. The protocol must have perfect completeness, the knowledge error must be $1/q$, and it should satisfy the honest-verifier zero-knowledge property.
    \end{question}
    \tcblower{}
    \begin{proof}
    In an honest verifier setting, the zero-knowledge proof-of-knowledge can be defined as follows: 
    \begin{itemize}
        \item Consider an honest verifier $\calV$ and a prover $\calP$.
        \item \textbf{Common input:} $g, u, h, v$ where all of them are group elements.
        \item \textbf{Prover's private input:} $a$ s.t. $u = g^a , v = h^a$
        \item \textbf{Claim to prove:} Prover knows a
    \end{itemize}

    \textbf{The protocol:}
    \begin{enumerate}
        \item \textbf{P:} prover picks random $t_1, t_2, t_3$ and sends $x = g^{t_1}$, $y = h^{t_2}$ and $z = (g \cdot h)^{t_3}$ to the verifier
        \item \textbf{V:} verifier sends uniformly random $c_1, c_2, c_3$ with respect to the pairs $(g,u),(h,v),(g \cdot h,u \cdot v)$
        \item \textbf{P:} prover calculates $w_1 = t_1 +c_1\cdot a$, $w_2 = t_2 +c_2\cdot a$ and $w_3 = t_3 +c_3\cdot a$ and sends $w_1, w_2, w_3$ to the verifier
        \item \textbf{V:} verifier checks if $x \cdot u^{c_1} = g^{w_1}$, $y \cdot v^{c_2} = h^{w_2}$ and $z \cdot (u \cdot v)^{c_3} = (g \cdot h)^{w_3}$
    \end{enumerate}

    \textbf{Completeness:}\\
        For every tuple $(w_i,t_i,c_i)$ given prover knows the value a
        \begin{equation}
        \begin{split}
                x\cdot u^{c_1} = g^{t_1}*g^{a\cdot c_1} &= g^{t_1+a\cdot c_1} = g^{w_1}\\
                y\cdot v^{c_2} = h^{t_2}*g^{a\cdot c_2} &= h^{t_2+a\cdot c_2} = h^{w_2}\\
                z\cdot (u\cdot v)^{c_3} = (g\cdot h)^{t_3}*(g\cdot h)^{a\cdot c_3} &= (g\cdot h)^{t_3+a\cdot c_3} = (g\cdot h)^{w_3}\\   
        \end{split}
        \end{equation}
        Hence an honest verifier is always convinced for an honest prover, perfect completeness\\
    \newpage

    \textbf{Knowledge error:}\\
    We define the extractor's run as follows: 
    \begin{enumerate}
        \item Extractor receives $x,y,z$
        \item It then sends uniformly random $c_1$, $c_2$, $c_3$ to the prover
        \item Then it receives $w_1$, $w_2$, $w_3$ in response
        \item Extractor rewinds upto after point 1
        \item It sends uniformly random $c_1'$, $c_2'$, $c_3'$ and receives $w_1'$, $w_2'$, $w_3'$
        \item The extractor outputs $(w_1'-w_1)/(c_1'-c_1)$ which is equal to $(w_2'-w_2)/(c_2'-c_2)$ and $(w_3'-w_3)/(c_3'-c_3)$ assuming prover has the correct witness
    \end{enumerate}
    %I am dumb, pls help
    probability of cheating = a chosen by prover is accidentally the witness => 1/q\\

    % simulator:
    % simply feed some randomness and send results accordingly
    \textbf{HVZK:}\\
    We can define the honest verifier zero-knowledge simulator as follows:
    \begin{enumerate}
        \item Simulator picks $c_1, c_2, c_3 \leftarrow \z_q$ and $w_1, w_2, w_3 \leftarrow \z_q$
        \item it sets $x = g^w_1/u^c_1$, $y = h^w_2/v^c_2$, $z = (g\cdot h)^w_3/(u\cdot v)^c_3$
        \item It then outputs the tuples $(x,c_1,w_1), (y,c_2,w_2)$ and $(z,c_3, w_3)$
    \end{enumerate}
    This transcript is identical to the real-world transcript. Hence we have a simulator.
    \end{proof}
\end{solution}
