\begin{solution}{ZK protocol for DDH}\label{ques:23}
    \begin{question}
    Next, consider the complement of $\lddh$, denoted by $\lnddh$ (defined below).
    \begin{equation}
        \lnddh = \{(g,h,u,v): \forall a \in \z_q \text{ either } u \neq g^a \text{ or } v \neq h^a \}
    \end{equation}
    Construct a protocol for $\lnddh$. The protocol must have perfect completeness, constant soundness and it should satisfy zero-knowledge w.r.t. auxiliary information.
    \end{question}
    \tcblower{}
    \begin{proof}
    In a zero-knowledge with auxiliary information setting, the common input is the same as before. We have to show that the given prover can distinguish between the groups created by $u\cdot v$ and $g \cdot h$. The protocol for $\lnddh$ is defined as follows:
    \begin{enumerate}
        \item \textbf{V:} Verifier chooses a bit $b$ and $x\leftarrow \z_q$ and a message $m=g\cdot h$ when $b = 0$ and $m = u \cdot v$ when $b = 1$. Where $u$ and $v$ are defined as $g^x$ and $h^x$ respectively.
        \item \textbf{V:} It sends a zero-knowledge proof of knowledge to the prover that it has chosen a bit b and x. This is possible since verifying $\exists b : \exists x : m_b^x$ is possible in polynomial time ($O(1)$ time) and zero-knowledge reductions exist to NP-complete problem. This is taken as a preventive measure towards avoiding the verifier from using the auxiliary information.
        \item \textbf{V:} Verifier then sends $y = m_b^x$ to the prover
        \item \textbf{P:} Prover then sends a bit $b'$ by distinguishing if $m_b^x$ was a group element of $m_0$ or $m_1$
    \end{enumerate}
    \textbf{Completeness:}\\
    Since prover is unbounded, prover can find all the group elements for $u\cdot v$ and $g\cdot h$. It is a yes instance for $\lnddh$; therefore, both groups will be disjoint, and hence it can identify and send the appropriate bit $b' = b$. Hence it has perfect completeness.\\
    
    \textbf{Soundness:}\\
    In case the prover is not able to identify the bit $b$ for the message $y$, it can output the correct answer bit with the probability $\frac{1}{2}+\epsilon$ where $\epsilon$ is negligible information the prover can extract from the zero-knowledge proof provided by the verifier. \\
    
    \textbf{Simulator for ZK with auxiliary info:}
    \protocol{Simulator $\calS$}{Simulator for zero-knowledge protocol with auxiliary information}{22sim}{
            \begin{enumerate}
                \item Simulator extracts the witness from the verifier and obtains $m_b^x$ from the verifier. Simulator repeats this extraction atmost $n$ times until it gets a valid witness.
                \item If the simulator was able to extract a valid witness, it sends the obtained witness $b$ to the verifier and outputs the final transcript else it outputs $\perp$.
            \end{enumerate}
    }
    Probability of success of simulator (let probability of verifier choosing the bit $b = 0$ be $p$):
    \begin{equation}
        \begin{split}
            \prob{\text{simulator succeeds}} &= \prob{\text{simulator extracts successfully in atmost}n\text{ attempts}}\\
            &= 1 - \prob{\text{extractor fails}}^n\\
            &= 1 - (1 - (\epsilon - \kappa))^n   \\
            & = 1 - \mu(n)
        \end{split}
    \end{equation}
    $\epsilon - \kappa$ is non negligible since verifier sends a proof of knowledge. Therefore $1-(\epsilon - \kappa)<1$ by a non negligible amount. Hence after repeating (rewinding) for n iterations, the probability of the extractor outputting $\perp = (1-(\epsilon - \kappa))^n$ which after enough iterations becomes negligible. The argument works since the witness, if successfully extracted, will always be unique.\\
    
    The transcript outputted by the simulator looks identical to a valid prover's transcript. Hence we have a protocol for zero knowledge with auxiliary information.
    \end{proof}
\end{solution}
