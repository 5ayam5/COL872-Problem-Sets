\begin{solution}{Question 2: Collapsing hash from claw-free functions}\label{ques:x}
    \begin{question}
    Show that $\{H_k \equiv f_k\}_{k\in K}$ is a collapsing hash function.
    \end{question}
    \tcblower{}
    \begin{proof}
    We will Begin the proof by stating that the claw free function is a post-quantum CRHF. This can be said because For two bit-strings ($x_0,x_1$) in the domain of claw-free function, there can be two cases: \begin{enumerate}
        \item $x_0$ and $x_1$ have the same first bit. In this case no collision is possible between $x_0$ and $x_1$ as the claw-free function $f_k(b,\cdot)$ is injective $\forall k \in \calK, b \in \{0,1\}$
        \item $x_0$ and $x_1$ have different first bit. In this case as well there is no collision possible due to the security definition of claw free function.
    \end{enumerate}

    So In both the cases we have no q.p.t that can give a collision and thus claw free function is a collision resistant hash function.

    Now Let us assume that there exist a adversary A that breaks Collapsing property with probability $\rho$ for $\calH$ then a reduction B can finds a collision with prob poly($\rho$).

    Notations:
    \begin{itemize}
        \item A Set S which is a subset of the set U($\{0,1\}^{n+1}$) which has the same image over $f_k$ and has size l(for claw-free function l=2).
        \item Another set V
        \item A Superposition $\varphi$ over pairs (s,v) $\in S \times V$.
        \item A Binary Projective Measurement P = \textbf{(P,I-P)}
    \end{itemize}

    Adversary has two Algorithms $\calA_1$ and $\calA_2$ which are defined as follows:

    $\calA_1$: sends superposition $\varphi$.

    $\calA_2$: applies Measurement P and sends the mixed state obtained and bit b'.\newline

    The Reduction is as follows:
    \begin{enumerate}
        \item Challenger C sends a key $k\in \calK$ to B who passes the same to A.
        \item A applies Algorithm $\calA_1$ and sends superposition $\varphi$ to B.
        \item B applies projective measurement over $\varphi$ to get a value i and the state $\varphi'$.
        \item B sends $\varphi'$ to A.
        \item A applies Algorithm $\calA_2$ over $\varphi'$ and sends b' and the state $\varphi''$ to B. 
        \item B applies projective measurement over $\varphi$ to get a value j.
        \item B sends i,j to C.
        
    \end{enumerate}
    We will show that the probability of finding a Collision is $\geq \frac{2}{l-1}\rho^2$
    Now let us find the probability of finding a Collision, $Pr\Bigr[i,j \in S, i\neq j\Bigr]$.\newline

    We assume $\varphi = \ket{\psi}\bra{\psi}$ for some pure state $\ket{\psi} = \Sigma_{i,v}\alpha_{i,v}\ket{i,v}$\newline

    The Probability of finding i in the first projective measurement is $p_i = $Tr$[(\textbf{I}\otimes \ket{i}\bra{i})\varphi ]$. here $\varphi'$ becomes $\varphi_i := \frac{1}{p_i}(\textbf{I}\otimes \ket{i}\bra{i})\varphi(\textbf{I}\otimes \ket{i}\bra{i})$.\newline

    Now, When Algorithm $\calA_2$ applies P, the resulting mixed state becomes $\varphi'_i = \textbf{P}\varphi\textbf{P} + \textbf{(I-P)}\varphi\textbf{(I-P)}$. On applying projective measurement again, the probability of getting j is Tr$[(\textbf{I}\otimes \ket{j}\bra{j})\varphi'_i ]$. Now summing over all $i\in S$ and $j \in S/\{i\}$, we get the probability of getting distinct $i,j \in S$, which is
    \begin{equation*}
    \begin{split}
        Pr\Bigr[i,j \in S, i\neq j\Bigr] &= \text{Tr}\Bigg[\sum_{i,j\in S, i\neq j} \substack{(\textbf{I}\otimes \ket{j}\bra{j})\textbf{P}(\textbf{I}\otimes \ket{i}\bra{i})\varphi(\textbf{I}\otimes \ket{i}\bra{i})\textbf{P} \\+ (\textbf{I}\otimes \ket{j}\bra{j})\textbf{(I-P)}(\textbf{I}\otimes \ket{i}\bra{i})\varphi(\textbf{I}\otimes \ket{i}\bra{i})\textbf{(I-P)}}\Bigg]\\
        &= \text{2Tr}\Biggr[\sum_{i,j\in S, i\neq j} (\textbf{I}\otimes \ket{j}\bra{j})\textbf{P}(\textbf{I}\otimes \ket{i}\bra{i})\varphi(\textbf{I}\otimes \ket{i}\bra{i})\textbf{P}\Biggr]\\
        &= \text{2Tr} \Bigg[\sum\limits_{\substack{i,j\in S, i\neq j \\ v,v'\in V}} \alpha_{i,v}\alpha'_{i,v'}(\textbf{I}\otimes\ket{j}\bra{j})\textbf{P}(\ket{v}\bra{v'}\otimes\ket{i}\bra{i})\textbf{P}\Bigg]\\
        &= \Bigg[\sum\limits_{\substack{i,j\in S, i\neq j \\ v,v'\in V}} \alpha_{i,v}\alpha'_{i,v'} (\ket{v'}\ket{i})\textbf{P}(\textbf{I}\otimes\ket{j}\bra{j})\textbf{P}(\ket{v}\ket{i})  \Bigg]\\
        &= \Bigg[\sum\limits_{\substack{i,j\in S, i\neq j \\ v,v',v''\in V}} \alpha_{i,v}\alpha'_{i,v'}\bra{v',i}\textbf{P}\ket{v'',j}\bra{v'',j}\textbf{P}\ket{v,i} \Bigg]
    \end{split}
    \end{equation*}
    Now let's define a vector w as $w_{(i,j,v'')} := \Sigma_v \alpha_{i,v}\bra{v'',j}\textbf{P}\ket{v,i}$ where $i\neq j$, we have \newline Probability = $2 |w^2|$. 

    To find $\rho$, 
    \begin{equation*}
        \begin{split}
            \rho &= \text{Tr}[\textbf{P}\varphi] - \text{Tr}\Bigg[\textbf{P}\sum_{i\in S}(\textbf{I}\otimes \ket{i}\bra{i})\varphi(\textbf{I}\otimes \ket{i}\bra{i})\Bigg]\\
            &= \Bigg[\sum\limits_{\substack{i,j\in S \\ v,v'\in V}} \alpha_{i,v}\alpha_{j,v'}^\dag \bra{v',i}\textbf{P}\ket{v,j} - \sum\limits_{\substack{i\in S \\ v,v'\in V}} \alpha_{i,v}\alpha_{i,v'}^\dag \bra{v',i}\textbf{P}\ket{v,i}\Bigg]\\
            &= \Bigg[\sum\limits_{\substack{i,j\in S, i\neq j \\ v,v'\in V}} \alpha_{i,v}\alpha_{j,v'}^\dag \bra{v',i}\textbf{P}\ket{v,j} \Bigg]
        \end{split}
    \end{equation*}
    Now if we define x as vector $x_(i,j,v'') := \alpha_{j,v''}$, we have $\rho = x\cdot w$. Also,
    \begin{equation*}
        |x|^2 = \sum\limits_{\substack{i,j\in S,i\neq j \\ v''\in V}} |\alpha_{j,v''}|^2 = \sum\limits_{\substack{j\in S, v''\in V}} (l-1) |\alpha_{j,v''}|^2 = l-1
    \end{equation*}
    Therefore, by applying Cauchy-Schwartz Inequality, we have $|w|^2|x|^2 \geq |w\cdot x|^2$. From this we get the required probability for finding a Collision.

    Now as we know via security of claw-free function, no q.p.t adversary can find a collision, the probability of finding a collision is negligible and from this we can say that for the case of claw-free function,an adversary can break Collapsing property with only negligible probability and thus $\{H_k \equiv f_k\}_{k\in K}$ is a collapsing hash function.
    \end{proof}
\end{solution}
