\mathchardef\mhyphen="2D
\documentclass[11pt]{article}
\usepackage[english]{babel}
\usepackage{minted}
\usepackage{amsfonts}
\usepackage{amsmath}
\usepackage{amsthm}
\usepackage{graphicx}
\usepackage{subcaption}
\usepackage[hypcap=false]{caption}
\usepackage{booktabs}
\usepackage[left=25mm, top=25mm, bottom=25mm, right=25mm]{geometry}
\usepackage{soul}
\usepackage{algorithm}
\usepackage{algpseudocode}
\usepackage[most]{tcolorbox}
\usepackage[colorlinks=true,linkcolor=darkcyan,filecolor=darkcerulean,urlcolor=magenta]{hyperref}


\newcommand{\calA}{\mathcal{A}}
\newcommand{\calB}{\mathcal{B}}
\newcommand{\calC}{\mathcal{C}}
\newcommand{\calD}{\mathcal{D}}
\newcommand{\calE}{\mathcal{E}}
\newcommand{\calF}{\mathcal{F}}
\newcommand{\calG}{\mathcal{G}}
\newcommand{\calH}{\mathcal{H}}
\newcommand{\calI}{\mathcal{I}}
\newcommand{\calJ}{\mathcal{J}}
\newcommand{\calK}{\mathcal{K}}
\newcommand{\calL}{\mathcal{L}}
\newcommand{\calM}{\mathcal{M}}
\newcommand{\calN}{\mathcal{N}}
\newcommand{\calO}{\mathcal{O}}
\newcommand{\calP}{\mathcal{P}}
\newcommand{\calQ}{\mathcal{Q}}
\newcommand{\calR}{\mathcal{R}}
\newcommand{\calS}{\mathcal{S}}
\newcommand{\calT}{\mathcal{T}}
\newcommand{\calU}{\mathcal{U}}
\newcommand{\calV}{\mathcal{V}}
\newcommand{\calW}{\mathcal{W}}
\newcommand{\calX}{\mathcal{X}}
\newcommand{\calY}{\mathcal{Y}}
\newcommand{\calZ}{\mathcal{Z}}

\newcommand{\bfA}{\mathbf{A}}
\newcommand{\bfB}{\mathbf{B}}
\newcommand{\bfC}{\mathbf{C}}
\newcommand{\bfD}{\mathbf{D}}
\newcommand{\bfE}{\mathbf{E}}
\newcommand{\bfS}{\mathbf{S}}

\newcommand{\N}{\mathbb{N}}
\newcommand{\z}{\mathbb{Z}}
\newcommand{\I}{\mathbb{I}}

\newcommand{\keygen}{\mathsf{KeyGen}}
\newcommand{\enc}{\mathsf{Enc}}
\newcommand{\dec}{\mathsf{Dec}}
\newcommand{\negl}{\mathsf{negl}}

\newcommand{\setup}{\mathsf{Setup}}
\newcommand{\lsetup}{\mathsf{Setup\text{-}Lossy}}

\newcommand{\eval}{\mathsf{Eval}}
\newcommand{\obf}{\mathsf{Obf}}

\newcommand{\phybb}[1]{p_{\mathrm{hyb}, #1}}
\newcommand{\lin}{\ell_{\mathrm{in}}}
\newcommand{\lout}{\ell_{\mathrm{out}}}
\newcommand{\bit}{\{0,1\}}
\newcommand{\sd}{\mathsf{SD}}
\newcommand{\swap}{\mathsf{swap}}
\newcommand{\lwe}{\mathsf{LWE}_{n,m,q,\chi}}
\newcommand{\sslwe}{\mathsf{ss\text{-}LWE}_{n,m,q,\chi}}

\newcommand{\unif}[1]{\mathsf{Unif}[-#1, #1]}
\newcommand{\func}[2]{\mathsf{Func}[#1, #2]}
\newcommand{\perm}[2]{\mathsf{Perm}[#1, #2]}
\newcommand{\ct}{\mathsf{ct}}
\newcommand{\Finverse}{F^{-1}}

\newcommand{\nqss}{\mathsf{No\mhyphen Query \mhyphen Semantic \mhyphen Security}}

\newcommand{\prob}[1]{\Pr\left[ #1 \right]}

\newlength{\protowidth}
\newcommand{\pprotocol}[4]{
{\begin{center}
\setlength{\protowidth}{\textwidth}
\addtolength{\protowidth}{-3\intextsep}

\fbox{
        \small
        \hbox{\quad
        \begin{minipage}{\protowidth}
    \begin{center}
    {\bf #1}
    \end{center}
        #4
        \end{minipage}
        \quad}
        }
        \captionof{figure}{\label{#3} #2}
\end{center}
} }

\newcommand{\defbox}[1]{
{\begin{center}
\setlength{\protowidth}{\textwidth}
\addtolength{\protowidth}{-3\intextsep}

\fcolorbox{darkcerulean}{cottoncandy}{
        \small
        \hbox{\quad
        \begin{minipage}{\protowidth}
    
        #1
        \end{minipage}
        \quad}
        }
\end{center}
        } }

\newcommand{\protocol}[4]{
\pprotocol{#1}{#2}{#3}{#4} }

\newtheorem{theorem}{Theorem}[section]
\newtheorem{claim}[theorem]{Claim}
\newtheorem{fact}[theorem]{Fact}
\newtheorem*{question}{Question}

\newtcolorbox{solution}[2][]{every float=\centering,breakable,enhanced,adjusted title={#2},colback=codegray,colframe=codegray!50!black}

\linespread{1.0}

\definecolor{codegray}{rgb}{0.98,0.97,0.93}
\definecolor{cottoncandy}{rgb}{1.0, 0.84, 0.95}
\definecolor{darkcerulean}{rgb}{0.03, 0.27, 0.49}
\definecolor{darkcyan}{rgb}{0.0, 0.50, 0.45}

\title{COL872\\Problem Set 1}
\author{Mallika Prabhakar (2019CS50440)\\Sayam Sethi (2019CS10399)\\Satwik Jain (2019CS10398)}
\date{January 2023}

\begin{document}

\maketitle

\tableofcontents

\pagenumbering{arabic}

\newpage
\section{Question 1}
\begin{solution}{Question 1}\label{ques:1}
    \begin{question}
    quess
    \end{question}
    \tcblower{}
    \begin{proof}
    \end{proof}
\end{solution}



\newpage
\section{Question 2}
\begin{solution}{Question 2: A Lossy Encryption Scheme based on LWE}\label{ques:2}
    \begin{question}
    In this problem, you will have to construct a lossy encryption mode for Regev
encryption. The algorithms $\setup, \enc, \dec$ are defined as in class (see Lecture Notes).
You must define the $\lsetup$ algorithm, and then show that it is a secure lossy
encryption scheme.
    \end{question}
    \tcblower{}
    \begin{proof}
    this proof
    \end{proof}
\end{solution}



\newpage
\section{Question 3}
\subsection{Question 3.1}
\begin{solution}{Question 1}\label{ques:1}
    \begin{question}
    quess
    \end{question}
    \tcblower{}
    \begin{proof}
    \end{proof}
\end{solution}


\subsection{Question 3.2}
\begin{solution}{Question 1}\label{ques:1}
    \begin{question}
    quess
    \end{question}
    \tcblower{}
    \begin{proof}
    \end{proof}
\end{solution}


\subsection{Question 3.3}
\begin{solution}{Question 3.3: Small Secrets LWE - Matrix Version}\label{ques:33}
    \begin{question}
    Recall, in the Learning-with-errors problem, if $m = n$, then the two distributions are statistically indistinguishable. Does the same hold true in the small- secrets setting?
    \end{question}
    \tcblower{}
    \begin{proof}
    As Small Secrets LWE and normal LWE are equivalent to each other,if $m=n$, the small secrets lwe will maintain statistical indistinguishability. Otherwise as there is a polynomial time reduction from $\sslwe$ to $\lwe$, then if $\sslwe$ is distinguishable, then it can be reduced to a $\lwe$ which will also be distinguishable, which contradicts the established fact that $\lwe$ is statistically indistinguishable, if $m=n$.
    \end{proof}
\end{solution}


\newpage
\section{Question 4}
\begin{solution}{Question 1}\label{ques:1}
    \begin{question}
    quess
    \end{question}
    \tcblower{}
    \begin{proof}
    \end{proof}
\end{solution}



\newpage
\section{Question 5}
\begin{solution}{Question 5: xyz}\label{ques:5}
    \begin{question}
    ques
    \end{question}
    \tcblower{}
    \begin{proof}
    this proof
    \end{proof}
\end{solution}



\newpage
\section{Question 6}
\subsection{Question 6.1}
\begin{solution}{Question 1}\label{ques:1}
    \begin{question}
    quess
    \end{question}
    \tcblower{}
    \begin{proof}
    \end{proof}
\end{solution}


\subsection{Question 6.2, 6.3, 6.4}
\begin{solution}{Question 1}\label{ques:1}
    \begin{question}
    quess
    \end{question}
    \tcblower{}
    \begin{proof}
    \end{proof}
\end{solution}


\end{document}
