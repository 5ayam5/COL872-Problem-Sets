\begin{solution}{Question 5.3}\label{ques:5.3}
    \begin{question}
    Suppose we encrypt a quantum state $\ket{\psi}$ using a random 2-bit key ( a, b). The resulting density matrix is\\
$ \rho = \frac{1}{4} \ket{\psi}\bra{\psi}+\bfX\ket{\psi}\bra{\psi}\bfX+\bfZ\ket{\psi}\bra{\psi}\bfZ+\bfX\cdot\bfZ\ket{\psi}\bra{\psi}\bfZ\cdot\bfX$
Using the above two parts, what can we conclude about $\rho$?
    \end{question}
    \tcblower{}
    \begin{proof}
    Consider the density matrix of $\ket{\psi}$. It can be written as a linear combination of matrices in $\mathcal{M}$,
    \begin{equation}
        \begin{split}
            \ket{\psi}\bra{\psi} &= \alpha_1\bfI + \alpha_2\bfX + \alpha_3\bfZ + \alpha_4\bfX\cdot\bfZ\\
                                 &= \begin{bmatrix}
                                        \alpha_1 + \alpha_3&\alpha_2-\alpha_4\\
                                        \alpha_2 + \alpha_4&\alpha_1 - \alpha_3
                                    \end{bmatrix}
        \end{split}
    \end{equation}
    From the fact that the density matrices have a trace $= 1$, we get that $\alpha_1 = 1/2$. Now consider the resultant density matrix after encrypting $\ket{\psi}$,
    \begin{equation}
        \begin{split}
            \rho &= \frac{1}{4} \ket{\psi}\bra{\psi}+\bfX\ket{\psi}\bra{\psi}\bfX+\bfZ\ket{\psi}\bra{\psi}\bfZ+\bfX\cdot\bfZ\ket{\psi}\bra{\psi}\bfZ\cdot\bfX\\
                 &= f(\ket{\psi})\\
                 &= f(\alpha_1\bfI + \alpha_2\bfX + \alpha_3\bfZ + \alpha_4\bfX\cdot\bfZ)\\
                 &= \alpha_1 f(\bfI) + \alpha_2 f(\bfX) + \alpha_3 f(\bfZ) + \alpha_4 f(\bfX\cdot \bfZ)\text{ (since $f$ is a linear operator)}\\
                 &= \alpha_1 f(\bfI) = \frac{1}{2}\bfI
        \end{split}
    \end{equation}
    Therefore, we get that the resultant density matrix is independent of $\ket{\psi}$ and therefore the proposed encryption scheme perfectly hides the qubit.
    \end{proof}
\end{solution}
