\begin{solution}{Question 6.1}\label{ques:61}
    \begin{question}
        Let $\rho$ be the density matrix for a mixed state over $n$ qubits. In class, we saw that there exists a pure state $\ket{\psi}$ over $2n$ qubits such that measuring the last $n$ qubits results in the density matrix $\rho$. Using Schmidt decomposition, prove that if $\ket{\psi_1}$ and $\ket{\psi_2}$ are two purifications of $\rho$, then there exists a unitary matrix $\mathbf{U}$ acting over $n$ qubits such that $\ket{\psi_2} = (\mathbf{I}_n \otimes \mathbf{U}) \ket{\psi_1}$. Here $\mathbf{I}_n$ is the identity operation over the first $n$ qubits.
    \end{question}
    \tcblower{}
    \begin{proof}
        Let,
        \begin{equation}
            \begin{split}
                \ket{\psi_1} &= \sum_i\lambda_{1i}\ket{u_{1i}}\ket{v_{1i}}\\
                \ket{\psi_2} &= \sum_i\lambda_{2i}\ket{u_{2i}}\ket{v_{2i}}
            \end{split}
        \end{equation}
        On measuring the last $n$ qubits, we are left with,
        \begin{equation}
            \begin{split}
                \rho = \rho_1 &= \sum_i\lambda_{1i}^2\ket{u_{1i}}\bra{u_{1i}}\\
                =\rho_2 &= \sum_i\lambda_{2i}^2\ket{u_{2i}}\bra{u_{2i}}
            \end{split}
        \end{equation}
        Since $\{\ket{u_{1i}}\}_i$ and $\{\ket{u_{2i}}\}_i$ are orthonormal vectors, the two multi-sets $\{\lambda_{1i}\}_i$ and $\{\lambda_{2i}\}_i$ should be the same and they form the eigenvalues of $\rho$. Therefore, we can assume the ordering of $\{\ket{u_{1i}}\}_i$ and $\{\ket{u_{2i}}\}_i$ such that $\lambda_{1i} = \lambda_{2i} = \lambda_i$ (direct equality holds since $\lambda_{bi}$ are guaranteed to be positive by Schmidt decomposition).\par
        Now, we represent $\ket{\psi_2}$ such that the first $n$ qubits have the same orthonormal vectors as $\ket{\psi_1}$. It is guaranteed that $\ket{u_{1i}} = \ket{u_{2i}}$ if the multiplicity of $\lambda_i^2$ is $1$. Consider $\lambda_p^2$ such that it has a multiplicity $k > 1$. The two sets of eigenvectors corresponding to this eigenvalue are $S_1 = \{\ket{u_{1i}} | \lambda_i = \lambda_p\}$ and $S_2 = \{\ket{u_{2i}} | \lambda_i = \lambda_p\}$. Now, these two sets span the same subspace of $n$ qubits. Therefore, we can write $\sum_{\ket{u_{2i}}\in S_2}\ket{u_{2i}}\ket{v_{2i}}$ as,
        \begin{equation}
            \begin{split}
                \sum_{\ket{u_{2i}}\in S_2}\ket{u_{2i}}\ket{v_{2i}} &= \sum_{\ket{u_{2i}}\in S_2}\left(\sum_{\ket{u_{1j}}\in S_1}\alpha_{ij}\ket{u_{1j}}\right)\ket{v_{2i}}\\
                &= \sum_{\ket{u_{2i}}\in S_2}\left(\sum_{\ket{u_{1j}}\in S_1}\ket{u_{1j}}\alpha_{ij}\ket{v_{2i}}\right)\\
                &= \sum_{\ket{u_{1j}}\in S_1}\ket{u_{1j}}\left(\sum_{\ket{u_{2i}}\in S_2}\alpha_{ij}\ket{v_{2i}}\right)\\
                &= \sum_{\ket{u_{1j}}\in S_1}\ket{u_{1j}}\ket{v_{2j}'}
            \end{split}
        \end{equation}
        Note that $\{\ket{v_{2j}'} | \lambda_j = \lambda_p\}$ is an orthonormal set since $S_2$ is also an orthonormal set. Therefore, $\ket{\psi_2}$ can be written as,
        \begin{equation}
            \ket{\psi_2} = \sum_i\lambda_i\ket{u_{1i}}\ket{v_{2i}'} = \sum_i\lambda_i\ket{u_i}\ket{v_{2i}'}
        \end{equation}
        Now, since $\{\ket{v_{1i}}\}_i$ and $\{\ket{v_{2i}'}\}_i$ are both orthonormal sets, there exists a change of basis matrix (assuming that both sets span the entire set of $n$ qubits, else, we can extend them to span the entire set), say $\mathbf{U}$. Therefore, we can write $\ket{\psi_2}$ in terms of $\ket{\psi_1}$ as,
        \begin{equation}
            \ket{\psi_2} = (\mathbf{I}_n\otimes \mathbf{U})\ket{\psi_1}
        \end{equation}
        This completes the proof.
    \end{proof}
\end{solution}
