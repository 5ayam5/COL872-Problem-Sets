\begin{solution}{Question 4}\label{ques:4}
    \begin{question}
    Design an efficient quantum algorithm for the ‘Modified Simon’s Problem’. Here, the algorithm is given oracle access to two functions $f_1: \{0, 1\}^n \rightarrow \{0, 1\}^n$, $f_2: \{0, 1\}^n \rightarrow \{0, 1\}^n$ with the guarantee that there exist a bit-string $s \in \{0, 1\}^n$ such that for all $x \in \{0, 1\}^n$, $f_1(x) = f_2(x \oplus s)$. The algorithm must output $s$ with non-negligible probability, and can make $poly(n)$ queries to $f_1$, $f_2$. Describe the quantum circuit using unitaries $U_{f_1}$,$U_{f_2}$.
    \end{question}
    \tcblower{}
    \begin{proof}
    The quantum circuit using the unitaries $U_{f_1}$ and $U_{f_2}$ can be described as follows:\\
    \begin{quantikz}[thin lines] 
            \lstick{$\ket{0^n}$} & \gate{H^{\otimes n}} & \gate[wires = 2]{U_{f_1}} & \gate[wires = 2]{U_{f_2}} & \gate{H^{\otimes n}} & \meter{} & \qw \\
            \lstick{$\ket{0^n}$}  &\qw & \qw & &  \qw & \meter{} & \qw       
    \end{quantikz}

    The quantum algorithm must output s with a non-negligible probability given oracle access with poly(n) queries to $f_1$ and $f_2$. We will show an efficient quantum algorithm for this problem by reducing it to the original Simon's Problem.


    \begin{claim}\label{claim:4.1}
    There exists an equivalent problem to solve as Simon's problem.
    \end{claim}

    \begin{proof}
        Construct a function $g : \{0,1\}^n \rightarrow \{0,1\}^n$ such that $g(x) = f_1(x)\oplus f_2(x)$\\
        Now, let us look at $g(x\oplus s)$ where $s \in \{0, 1\}^n$  is a bit string s.t. for all $x \in \{0, 1\}^n$, $f_1(x) = f_2(x \oplus s)$.
        \begin{equation}
            g(x \oplus s) = f_1(x \oplus s) \oplus f_2(x \oplus s)
            = f_2(x) \oplus f_1(x)
            = g(x)
        \end{equation}
        Here, $g(x\oplus s)$ will be equal to $g(x)$ only when $f_1(x) = f_2(x \oplus s)$ holds for all $x$. Hence Modified Simon's Problem can be reduced to Simon's problem.
        The oracle output for $g$ will be oracle output of $f_1 \oplus$ oracle output of $f_2$

    \end{proof}

    \begin{claim} \label{claim:4.2}
        $U_{f_1}$ followed by $U_{f_2}$ is same as the unitary $U_g$
    \end{claim}

    \begin{proof}
    Consider a quantum state $\ket{\psi_{in}} = \sum_{x,y} \alpha_{x,y}\ket{x}\ket{y}$. After $U_{f_1}$, we get $\ket{\psi_1} = \sum_{x,y} \alpha_{x,y}\ket{x}\ket{y \oplus f_1(x)}$. On applying $U_{f_2}$, we get $\ket{\psi_{out}} = \sum_{x,y} \alpha_{x,y}\ket{x}\ket{y \oplus f_1(x) \oplus f_2(x)}$\\

    Now for $U_g$, if initial quantum state $\ket{\psi_{in}} = \sum_{x,y} \alpha_{x,y}\ket{x}\ket{y}$, output quantum state would be $\ket{\psi_{out}} = \sum_{x,y} \alpha_{x,y}\ket{x}\ket{y \oplus g(x)} =  \sum_{x,y} \alpha_{x,y}\ket{x}\ket{y \oplus f_1(x) \oplus f_2(x)}$\\

    Given the same input states, output states for both are the same, hence applying $U_{f_1}$ after $U_{f_2}$ is the same as the unitary $U_g$.
    \end{proof}

    Claim \ref{claim:4.1} and \ref{claim:4.2} clearly show the efficient algorithm for finding a solution to the modified Simon's problem by successfully reducing it to Simon's Problem.
    
    \end{proof}
\end{solution}
